%%%%%%%%%%%%%%%%%%%%%%%%%%%%%%%%%%%%%%%%%%%%%%%%%%%%%%%%%%%%%%%%%%%%%%%%%%%%%%%%
\section*{\centering Part I. Informe Tecnico}
%%%%%%%%%%%%%%%%%%%%%%%%%%%%%%%%%%%%%%%%%%%%%%%%%%%%%%%%%%%%%%%%%%%%%%%%%%%%%%%%

\section{Introducció i Objectius}

Aquesta secció ha d’incloure una breu descripció del problema que es vol resoldre i la seva motivació. Cal contextualitzar el projecte dins del marc de l'assignatura i de les aplicacions pràctiques de les estructures de dades lexicogràfiques. També s’hi han d’enumerar els objectius principals del treball.

\section{Antecedents}

Aquesta secció presenta els coneixements previs i el marc teòric necessari per entendre el projecte (Background). Es poden descriure aquí les estructures clàssiques de tries, les seves variants (radix tries, tries compactes, etc.), les seves aplicacions i propietats asimptòtiques. També és útil incloure una revisió breu de literatura o referències a llibres de text o articles que han estat útils per al disseny del projecte.

\section{Disseny i Implementació}

En aquest apartat cal descriure amb detall les decisions de disseny i les implementacions realitzades. Cal explicar quines estructures s’han utilitzat, com s’han representat internament i quines operacions s’han implementat (com ara inserció, cerca, autocompletat, etc.), tot destacant com s’ha garantit l’eficiència de cadascuna. També és convenient discutir les possibles alternatives que s’han considerat durant el desenvolupament, així com justificar les opcions escollides. Finalment, s’hauria d’incloure una comparació qualitativa entre les diferents implementacions desenvolupades, tot valorant-ne avantatges, limitacions i aplicabilitat en funció de l’escenari o conjunt de dades.

\section{Avaluació Experimental}

Aquí cal presentar l’anàlisi empírica del rendiment de les estructures implementades. Es poden descriure els conjunts de dades emprats, les proves realitzades, les mètriques recollides (temps d’execució, memòria, nodes visitats, profunditat mitjana, etc.) i mostrar els resultats obtinguts de manera gràfica o tabulada. Finalment, s’hauria de fer una interpretació crítica dels resultats, contrastant-los amb les expectatives teòriques.

\section{Conclusions}

En aquesta secció s’han de resumir els principals resultats i aportacions del projecte. Es pot fer una valoració del rendiment assolit, de les dificultats trobades i del coneixement adquirit. També és pertinent incloure possibles extensions, millores o línies futures de treball si el projecte es volgués ampliar. Aquesta secció hauria de tancar el document de forma clara i reflexiva.
