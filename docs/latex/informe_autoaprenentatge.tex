%%%%%%%%%%%%%%%%%%%%%%%%%%%%%%%%%%%%%%%%%%%%%%%%%%%%%%%%%%%%%%%%%%%%%%%%%%%%%%%%
\section*{\centering Part II. Informe d'Autoaprenentatge}
%%%%%%%%%%%%%%%%%%%%%%%%%%%%%%%%%%%%%%%%%%%%%%%%%%%%%%%%%%%%%%%%%%%%%%%%%%%%%%%%

\section{Descripció i valoració del procés d'autoaprenentatge}

Aquesta part del document complementa l’informe tècnic i reflecteix el vostre procés personal i col·lectiu d’aprenentatge durant el desenvolupament del projecte. L’objectiu és analitzar com heu organitzat el treball, què heu après, quines dificultats heu superat i com heu crescut com a estudiants d'enginyeria.

\subsection{Metodologia}

Descriviu quina ha estat la metodologia emprada per planificar, dividir i executar el projecte. Podeu explicar com heu organitzat el treball en equip (si escau), quines eines heu fet servir per col·laborar o documentar-vos, com heu estructurat el codi i les proves, i com heu abordat l’experimentació. També podeu comentar si heu seguit alguna estratègia iterativa, incremental, àgil o altres enfocaments.

\subsection{Valoració del procés d'autoaprenentatge}

Reflexioneu sobre els coneixements adquirits, tant teòrics com pràctics. Indiqueu quins aspectes us han resultat més difícils i com els heu resolt, i valoreu fins a quin punt heu assolit els objectius inicials. Podeu mencionar conceptes nous que heu descobert, habilitats tècniques que heu millorat (programació, anàlisi, visualització de dades, etc.) o competències transversals (comunicació, gestió del temps, treball en equip). Si heu canviat d’enfocament durant el projecte, expliqueu el perquè i què heu après en el procés.
